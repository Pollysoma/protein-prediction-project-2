\documentclass[aspectratio=169,12pt]{beamer}
\usetheme{metropolis}
\metroset{progressbar=foot}
\usepackage[utf8]{inputenc}
\usepackage{graphicx}
\usepackage{xcolor}
\usepackage{array}
\usepackage{tikz}

\title{Project 3b: Exploring Overtraining and Membership in ESM-2}
\author{Egor Rakcheev \and Timon Giess \and Ala Sleimi \and Alexander Nielsen \and Polina Yugantyseva}
\date{\today}

\setbeamertemplate{navigation symbols}{}
\setbeamertemplate{footline}[frame number]
\graphicspath{{./}{slides/}{group_photos/}{../}}

\begin{document}

\begin{frame}
  \titlepage
\end{frame}

\begin{frame}{Team}
  \centering
  \renewcommand{\arraystretch}{1.1}
  \begin{tabular}{>{\centering\arraybackslash}m{2.1cm} >{\centering\arraybackslash}m{2.1cm} >{\centering\arraybackslash}m{2.1cm}}
    \shortstack[c]{\includegraphics[width=1.9cm]{group_photos/Ala.png}\\{\scriptsize Ala Sleimi}\\{\tiny MS Informatics}} &
    \shortstack[c]{\includegraphics[width=1.9cm]{group_photos/timon.jpg}\\{\scriptsize Timon Giess}\\{\tiny MS Informatics}} &
    \shortstack[c]{\includegraphics[width=1.9cm]{group_photos/Alexander.jpg}\\{\scriptsize Alexander Nielsen}\\{\tiny MS Data Science}} \\
    \shortstack[c]{\includegraphics[width=1.9cm]{group_photos/Egor.jpg}\\{\scriptsize Egor Rakcheev}\\{\tiny MS Informatics}} &
    \shortstack[c]{\includegraphics[width=1.9cm]{group_photos/Polina.jpg}\\{\scriptsize Polina Yugantyseva}\\{\tiny BS Bio+Info}} &
    \shortstack[c]{\tikz{\node[draw=none,minimum width=1.9cm,minimum height=1.9cm]{};}\\{\scriptsize ~}\\{\tiny ~}} \\
  \end{tabular}
\end{frame}

\begin{frame}{Membership attack recipe}
  \begin{itemize}
    \item Training-time proxy: sequences sampled from $\sim$ UniRef50. 
    \item Post-train proxy:  newly discovered proteins ($N$)
    \item Goal: learn a detector that flags members vs. non-members
  \end{itemize}
\end{frame}



\begin{frame}{Results snapshot}
  \begin{center}
    \includegraphics[width=0.78\linewidth]{great_graph.jpg}
  \end{center}
  \vspace{-0.2cm}
\end{frame}

\begin{frame}{Great job! 0.99 AUC!}
  \begin{center}
    \includegraphics[width=0.75\linewidth]{everyhing_is_fine.jpg}
  \end{center}
  \vspace{-0.15cm}
 
\end{frame}

\begin{frame}{Meta UR50D process (ESM-2 original)}
  \begin{itemize}
    \item Meta samples 43M UniRef50 clusters, then pulls UniRef90 sequences within clusters; $\sim$65M unique sequences seen from 138M \footnote{\href{https://doi.org/10.1126/science.ade2574}{Zeming Lin et al., \emph{Evolutionary-scale prediction of atomic-level protein structure with a language model}. Science 379, 1123--1130 (2023). DOI:10.1126/science.ade2574}.}.
    \item The exact UR50D training file was never released; only the sampling recipe is public.
  \end{itemize}
\end{frame}

\begin{frame}{What does that mean for MI?}
  \begin{itemize}
    \item For any candidate protein, only $\approx 65/138$ chance it was actually in the ESM-2 train pool.
    \item Our MI probes risk targeting ``maybe-seen'' sequences $\Rightarrow$ unclear ground truth.
    \item What were we even measuring??
  \end{itemize}
  \vspace{0.2cm}
  \begin{center}
    \includegraphics[width=0.22\linewidth]{they_trust_me.jpg}
  \end{center}
\end{frame}

\begin{frame}{NVIDIA to the rescue}
  \begin{itemize}
    \item Member proxy: NVIDIA UR50D train shards (T) --- mirrors Meta recipe but reproducible/public.
    \item Non-member proxies:
      \begin{itemize}
        \item New proteins ($N$) outside the train stream,
        \item Validation shards (V) from NVIDIA split,
        \item hard homologs from UniRef90 $\setminus$ T for .
      \end{itemize}
  \end{itemize}
\end{frame}

\begin{frame}{Hardness ladder}
  \begin{itemize}
    \item Easy: $(S1, N)$ --- $S1$ from T (train shards), $N$ are new proteins;  $|S1|=|N|=1{,}237$.
    \item Medium: $(S2, S3)$ --- $S2$ from T vs. $S3$ from V (val shards); equal size $K\!\approx\!12$k.
    \item Hard: $(S4, U4)$ --- $S4$ from T vs. $U4$ hard homologs from UniRef90 $\setminus$ T; 
  \end{itemize}

  Each of the 3 pairs is then split 80/20 into train/test for the MI detector. 
\end{frame}

\begin{frame}{Sampling}
  \begin{itemize}
    \item Bin by $(\text{Kingdom}, \text{Length bin})$ on $N$.
    \item All sampled sets mirror the empirical $N$ distribution to avoid easy shortcuts.
    \item Cluster-aware sampling (UR50) keeps representation balanced  
  \end{itemize}
\end{frame}

\begin{frame}{N distribution (kingdom + length)}
  \begin{center}
    \includegraphics[width=0.9\linewidth]{slides/n_distribution_combined.png}
  \end{center}
\small We use this empirical $N$ distribution to set bin targets across T/V and UniRef90 sampling.
\end{frame}



\begin{frame}{Accuracy (AUC)}

      \includegraphics[width=\linewidth]{outputs/plots/auc.png}
  
\end{frame}

\begin{frame}{ROC curves}
  \begin{center}
    \includegraphics[width=0.32\linewidth]{outputs/plots/roc_easy.png}\hfill
    \includegraphics[width=0.32\linewidth]{outputs/plots/roc_medium.png}\hfill
    \includegraphics[width=0.32\linewidth]{outputs/plots/roc_hard.png}
  \end{center}
  \small Left: N vs S1 (easy). Middle: S2 vs S3 (medium). Right: S4 vs U4 (hard).
\end{frame}

\begin{frame}{Idea from last week}
  \begin{itemize}
    \item Results were not promising even on flawed data.
    \item 650M embeddings are not ready yet (NVIDIA released 8M, 650M, 3B) so we still didn’t try them.
    \item Likely a waste of time.
  \end{itemize}
  \vspace{0.2cm}
  \begin{center}
    \includegraphics[width=0.18\linewidth]{group_photos/Ala.png}\\
    {\scriptsize sorry}
  \end{center}
\end{frame}

\begin{frame}{Biggest success \& next steps}
  \textbf{Biggest success since last presentation}
  \begin{itemize}
    \item Improved the data curation process.
    \item Identified the issue with using the Meta model.
  \end{itemize}
  \vspace{0.2cm}
  \textbf{Next steps (why/how they help)}
  \begin{itemize}
    \item Finish the 650M embeddings to test larger capacity models.
    \item Try other MI methods and ideas from the first presentation.
    \item Improve the MLP model and explore additional approaches.
  \end{itemize}
\end{frame}

\end{document}
